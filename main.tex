%============================ MAIN DOCUMENT ================================
\PassOptionsToPackage{table}{xcolor}
\documentclass[
  a4paper,
  invert-title,
  titleimage-ratio=13
]{bfhpub}

\usepackage[french,english,main=ngerman]{babel}  % https://www.namsu.de/Extra/pakete/Babel.html
\LoadBFHModule{listings,terminal,boxes}
%---------------------------------------------------------------------------
% Documents paths
%---------------------------------------------------------------------------
\makeatletter
\def\input@path{{content/}}
%or: \def\input@path{{/path/to/folder/}{/path/to/other/folder/}}
\makeatother
%-----------------  Base packages     --------------------------------------
% Include Packages
\usepackage{amsmath}          % various features to facilitate writing math formulas
\usepackage{amsthm}           % enhanced version of latex's newtheorem
\usepackage{amsfonts}         % set of miscellaneous TeX fonts that augment the standard CM
\usepackage{amssymb}          % mathematical special characters

\usepackage{siunitx}

\usepackage{graphicx}         % integration of images
\usepackage{float}            % floating objects

\usepackage{caption}          % for captions of figures and tables
\usepackage{subcaption}       % for subcaptions in subfigures
\usepackage{cite}             % use bibtex
\usepackage{wrapfig}

\usepackage{exscale}          % mathematical size corresponds to textsize
\usepackage{multirow}         % multirow emables combining rows in tables
\usepackage{multicol}

\usepackage{longtable}
\usepackage{adjustbox}

%---------------------------------------------------------------------------
% Graphics paths
%---------------------------------------------------------------------------
\graphicspath{{pictures/}{figures/}}
%---------------------------------------------------------------------------
% Blind text -> for dummy text
%---------------------------------------------------------------------------
\usepackage{blindtext}    
\usepackage{letltxmacro}   
\LetLtxMacro{\blindtextblindtext}{\blindtext}

\RenewDocumentCommand{\blindtext}{O{\value{blindtext}}}{
	\begingroup\color{BFH-Gray}\blindtextblindtext[#1]\endgroup
}
%---------------------------------------------------------------------------
% Glossary Package
%---------------------------------------------------------------------------
% the glossaries package uses makeindex
% if you use TeXnicCenter do the following steps:
%  - Goto "Ausgabeprofile definieren" (ctrl + F7)
%  - Select the profile "LaTeX => PDF"
%  - Add in register "Nachbearbeitung" a new "Postprozessoren" point named Glossar
%  - Select makeindex.exe in the field "Anwendung" ( ..\MiKTeX x.x\miktex\bin\makeindex.exe )
%  - Add this [ -s "%tm.ist" -t "%tmx.glg" -o "%tm.gls" "%tm.glo" ] in the field "Argumente"
%
% for futher informations go to http://ewus.de/tipp-1029.html
%---------------------------------------------------------------------------
\usepackage[nonumberlist]{glossaries-extra}
\makeglossaries
%----------------  Glossary Entries  ---------------------------------------
\newglossaryentry{latex}{
    name=LaTeX,
    description={Ein Textsatzsystem für wissenschaftliche Dokumente}
}

\newglossaryentry{bfh}{
    name=BFH,
    description={Berner Fachhochschule}
}

%----------------  Acronyms  -----------------------------------------------
\newacronym{ai}{AI}{Artificial Intelligence}
\newacronym{ml}{ML}{Machine Learning}
%---------------------------------------------------------------------------
% Makeindex Package
%---------------------------------------------------------------------------
\usepackage{makeidx}
\makeindex
%\usepackage{imakeidx}          % To produce index
%\makeindex[columns=2,intoc]    % Index-Initialisation
%\makeindex[columns=3,columnseprule,columnsep,intoc]
%---------------------------------------------------------------------------
% Hyperref Package (Create links in a pdf)
%---------------------------------------------------------------------------
\usepackage[
	,bookmarks
	,plainpages=false
	,pdfpagelabels
    ,pdfusetitle
	,backref = {false}          % No index backreference
	,colorlinks = {true}        % Color links in a PDF
	,hypertexnames = {true}     % no failures "same page(i)"
	,bookmarksopen = {true}     % opens the bar on the left side
	,bookmarksopenlevel = {0}   % depth of opened bookmarks
	,linkcolor=.
	,filecolor=.
	,urlcolor=.
	,citecolor=.
]{hyperref}
%---------------------------------------------------------------------------

%% %% Customize Footer and Headers in Document
%% \KOMAoptions{headsepline,plainheadsepline,footsepline,plainfootsepline}%
%% \setkomafont{headsepline}{\color{BFH-DarkBlue}}% BFH-DarkBlue required bfhcolors
%% \setkomafont{footsepline}{\color{BFH-DarkBlue}}%
%% \lehead*{lehead} % the * character does replace the header on the first chapter page as well
%% \cehead*{cehead}
%% \rehead*{rehead}
%% \lohead*{lohead}
%% \cohead*{cohead}
%% \rohead*{rohead}

%% \lefoot*{lefoot}
%% \cefoot*{cefoot}
%% \refoot*{refoot}
%% \lofoot*{lofoot}
%% \cofoot*{cofoot}
%% \rofoot*{rofoot}
%---------------------------------------------------------------------------
\begin{document}

%% Snippet do redefine German babel translations -- see FAQ latex.ti.bfh.ch for further information or
%% the package documentation https://www.ctan.org/pkg/translations
%% \redefinetranslation{German}{Advisor}{Dozentin}
%% \redefinetranslation{German}{advisor}{Dozentin} %  just for consistency
%% \redefinetranslation{German}{Author}{Autorin}
%% \redefinetranslation{German}{author}{Autorin} %  just for consistency
%% \redefinetranslation{German}{Co-advisor}{Mitbetreuerin}
%% \redefinetranslation{German}{co-advisor}{Mitbetreuerin} %  just for consistency
%% \redefinetranslation{German}{Expert}{Expertin}
%% \redefinetranslation{German}{expert}{Expertin} %  just for consistency
%% \redefinetranslation{German}{Project partner}{Projektpartnerin}
%% \redefinetranslation{German}{project partner}{Projektpartnerin} %  just for consistency
%------------ Translations -------------------------------------------------------------------------------------------------------------------------------
%------------ Additional commands ------------------------------------------------------------------------------------------------------------------------

%------------ START FRONT PART ---------------------------------------------------------------------------------------------------------------------------
\frontmatter % Nummerierung der Seiten in römischen Zahlen

%----------------  BFH tile page   -----------------------------------------
  \title{Projekt 4: Kombinatorische Logik}
  \subtitle{BTE5213 Laborprojekte, Herbstsemester 2025/2026\\Protokoll}
  \author{Janis Aebischer \and Simon Eisele}
  \department{Technik und Informatik}
  \institute{Elektrotechnik und Informationstechnologie}
  \version{1.0}
  \titlegraphic{\includegraphics{Bild1.png}}
  \maketitle

%------------ ABSTRACT        ----------------
\section{Abstract}
Das Abstract fasst die wesentlichen Inhalte einer wissenschaftlichen Arbeit konzise zusammen (ca. 150–250 Wörter). 
Es muss unabhängig von der Arbeit verständlich sein, denn es wird oft losgelöst von der Arbeit veröffentlicht. Es hilft dabei, sich einen groben Überblick über die Fragestellung, das Vorgehen und die zentralen Ergebnisse zu verschaffen. Bei wissenschaftlichen Artikeln entscheiden Interessierte anhand dieses Kurztexts oft darüber, ob sie den ganzen Beitrag lesen oder nicht. Ein
Abstract ist nicht bei jeder Arbeit verlangt.

%------------ TABLEOFCONTENTS ----------------
\tableofcontents

%------------ START MAIN PART ------------
\mainmatter

%------------ Introduction ------------
\clearpage
\section{Einleitung}
Das Ziel der Einleitung ist es, in das Thema des Versuchs einzuführen. In der Einleitung werden in erster Linie folgende Fragen beantwortet:
–	Worum geht es in diesem Versuch?
–	An welche theoretischen Themen knüpft dieser Versuch an?
–	Welches sind die Rahmenbedingungen? 
–	Die Aufgabenstellung für den Versuch [1] ist explizit zu referenzieren und im Literverzeichnis anzugeben.

\subsection{Vorbemerkungen zum Zielpublikum}
Die Adressaten sind angehende Ingenieure bzw. Ingenieurinnen, die keine Kenntnisse von dem Praktikumsversuch haben. Diese haben einen vergleichbaren Wissensstand wie die Autoren selbst. Das Protokoll soll diesem externen Zielpublikum einen einfachen Einstieg in den Versuch ermöglichen, auch ohne direkten Kontakt mit den Autoren zu haben.
Das Protokoll muss es der Leserschaft ermöglichen, einen Überblick über den Versuch und die erzielten Ergebnisse zu bekommen. Wie wurde an die Lösung herangegangen? Wie ist das entworfene System aufgebaut (hier ist typischerweise ein Blockdiagramm sehr hilfreich, das erläutert wird). Welche Funktionen erfüllen die Teilschaltungen? Welcher Stand wurde erreicht? Was funktioniert? Was funktioniert nicht? Wie wurde dies festgestellt?
Es muss dabei nicht alles bis ins letzte Detail beschrieben werden, aber die Lesenden müssen ein solches Verständnis des Versuchs erhalten, dass sie sich selbst einarbeiten können und die von Ihnen erzielten Ergebnisse mit vertretbarem Aufwand reproduzieren können.

\subsection{Hinweise}
Die Autorschaft für die einzelnen Abschnitte des Berichts sowie die erstellten Schaltungsblöcke in Logisim-evolution [2] müssen explizit angegeben werden, um eine individuelle Bewertung zu ermöglichen!
Nutzen Sie für eine konsistente Formatierung die definierten Formatvorlagen. Beispiele finden Sie im Anhang A. Vergessen Sie nicht für die finale Dokumentversion die grau hinterlegten Hinweise in dieser Protokollvorlage zu löschen und alle Verzeichnisse und Felder zu aktualisieren! Leere Verzeichnisse sind zu löschen!
Weitere Hinweise zum Erstellen technischer Berichte finden Sie, z.B., in [3], [4], [5] und [6]. Beachten Sie auch die Empfehlungen zur Verwendung und Deklaration KI-gestützter Tools im Modul BTE5213 [7]!

\clearpage
\section{Methoden und Materialien}

\clearpage
\section{Resultate}
Beschreiben Sie für jede Teilaufgabe Ihre Ergebnisse. Im Fall dieses Digitaltechnikversuches gehören dazu mindestens:
–	Besonderheiten der Lösung/Umsetzung
–	erstellte Wahrheitstabellen, KV-Diagramme und ermittelte Logikfunktionen
–	Abbildung der fertigen Schaltung für jeden Teilversuch und knappe Beschreibung Ihrer Struktur, Funktion und ggf. Besonderheiten. Referenzieren Sie jeweils explizit alle realisierten Logisim-Teilschaltungen mit ihren Schaltungsnamen und Name der Logisim-Datei.
–	definierte Testfälle und Ergebnisse der Validierung der Schaltungen in Logisim-evolution [2] und auf dem Leguan-Board [9] (Testprotokoll). Für komplexere Schaltungen mit vielen Eingangskombinationen ist die Auswahl der Testfälle zu begründen.
–	aufgetretene Probleme und deren Lösung

\subsection{BCD-zu-7-Segment-Decoder}
–	Vorbereitungsaufgabe
–	Typisierung der logischen Funktionen
–	Realisierung in Logisim-evolution
–	Testprotokoll

\subsection{Ripple-Carry-Addierer}
–	Volladdierer
–	8-Bit-Addierer
–	Testprotokoll

\subsection{Borrow-Bit-Subtrahierer}
–	Borrow-Subtrahierer
–	8-Bit-Subtrahierer
–	Testprotokoll

\subsection{Ergebnisumschalter}
–	Zwei-zu-eins 1-bit-Umschalter
–	Zwei-zu-eins 8-bit-Umschalter
–	Testprotokoll

\subsection{Binär-zu-BCD-Konverter}
–	Vorbereitungsaufgabe
–	Add3-Block
–	Gesamtschaltung
–	Testprotokoll

\clearpage
\section{Diskussion}
Ziel des Abschnitts Diskussion ist es die Ergebnisse des Versuchs zusammenzufassen und zu reflektieren.
–	Erreichte Funktionalität und verbleibende Probleme
–	Fazit und Reflexion
%------------ Authorship declaration translated to main language ------------
%\declarationOfAuthorship

%----------- Bibliography ----------------
\clearpage
\bibliographystyle{unsrt}
\bibliography{project}      % the project.bib file gets loaded

%------------ List of Figures ------------
\listoffigures
 
%------------ List of Tables -------------
\listoftables
 
%------------ List of Listings -----------
\lstlistoflistings 
 
%------------ Glossary -------------------
\printglossary

%------------ Index ----------------------
\clearpage
\printindex
%------------ Appendix ----------------	
\appendix
%\chapter{First Appendix Chapter}

\end{document}
