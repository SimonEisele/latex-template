Das Abstract fasst die wesentlichen Inhalte einer wissenschaftlichen Arbeit konzise zusammen (ca. 150–250 Wörter). 
Es muss unabhängig von der Arbeit verständlich sein, denn es wird oft losgelöst von der Arbeit veröffentlicht. Es hilft dabei, sich einen groben Überblick über die Fragestellung, das Vorgehen und die zentralen Ergebnisse zu verschaffen. Bei wissenschaftlichen Artikeln entscheiden Interessierte anhand dieses Kurztexts oft darüber, ob sie den ganzen Beitrag lesen oder nicht. Ein
Abstract ist nicht bei jeder Arbeit verlangt.