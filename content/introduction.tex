Das Ziel der Einleitung ist es, in das Thema des Versuchs einzuführen. In der Einleitung werden in erster Linie folgende Fragen beantwortet:
\begin{itemize}
    \item[-] Worum geht es in diesem Versuch?
    \item[-] An welche theoretischen Themen knüpft dieser Versuch an?
    \item[-] Welches sind die Rahmenbedingungen? 
    \item[-] Die Aufgabenstellung für den Versuch [1] ist explizit zu referenzieren und im Literverzeichnis anzugeben.
\end{itemize}

\subsection{Vorbemerkungen zum Zielpublikum}
Die Adressaten sind angehende Ingenieure bzw. Ingenieurinnen, die keine Kenntnisse von dem Praktikumsversuch haben. Diese haben einen vergleichbaren Wissensstand wie die Autoren selbst. Das Protokoll soll diesem externen Zielpublikum einen einfachen Einstieg in den Versuch ermöglichen, auch ohne direkten Kontakt mit den Autoren zu haben.
Das Protokoll muss es der Leserschaft ermöglichen, einen Überblick über den Versuch und die erzielten Ergebnisse zu bekommen. Wie wurde an die Lösung herangegangen? Wie ist das entworfene System aufgebaut (hier ist typischerweise ein Blockdiagramm sehr hilfreich, das erläutert wird). Welche Funktionen erfüllen die Teilschaltungen? Welcher Stand wurde erreicht? Was funktioniert? Was funktioniert nicht? Wie wurde dies festgestellt?
Es muss dabei nicht alles bis ins letzte Detail beschrieben werden, aber die Lesenden müssen ein solches Verständnis des Versuchs erhalten, dass sie sich selbst einarbeiten können und die von Ihnen erzielten Ergebnisse mit vertretbarem Aufwand reproduzieren können.

\subsection{Hinweise}
Die Autorschaft für die einzelnen Abschnitte des Berichts sowie die erstellten Schaltungsblöcke in Logisim-evolution [2] müssen explizit angegeben werden, um eine individuelle Bewertung zu ermöglichen!
Nutzen Sie für eine konsistente Formatierung die definierten Formatvorlagen. Beispiele finden Sie im Anhang A. Vergessen Sie nicht für die finale Dokumentversion die grau hinterlegten Hinweise in dieser Protokollvorlage zu löschen und alle Verzeichnisse und Felder zu aktualisieren! Leere Verzeichnisse sind zu löschen!
Weitere Hinweise zum Erstellen technischer Berichte finden Sie, z.B., in [3], [4], [5] und [6]. Beachten Sie auch die Empfehlungen zur Verwendung und Deklaration KI-gestützter Tools im Modul BTE5213 [7]!