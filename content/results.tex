Beschreiben Sie für jede Teilaufgabe Ihre Ergebnisse. Im Fall dieses Digitaltechnikversuches gehören dazu mindestens:
–	Besonderheiten der Lösung/Umsetzung
–	erstellte Wahrheitstabellen, KV-Diagramme und ermittelte Logikfunktionen
–	Abbildung der fertigen Schaltung für jeden Teilversuch und knappe Beschreibung Ihrer Struktur, Funktion und ggf. Besonderheiten. Referenzieren Sie jeweils explizit alle realisierten Logisim-Teilschaltungen mit ihren Schaltungsnamen und Name der Logisim-Datei.
–	definierte Testfälle und Ergebnisse der Validierung der Schaltungen in Logisim-evolution [2] und auf dem Leguan-Board [9] (Testprotokoll). Für komplexere Schaltungen mit vielen Eingangskombinationen ist die Auswahl der Testfälle zu begründen.
–	aufgetretene Probleme und deren Lösung

\subsection{BCD-zu-7-Segment-Decoder}
–	Vorbereitungsaufgabe
–	Typisierung der logischen Funktionen
–	Realisierung in Logisim-evolution
–	Testprotokoll

\subsection{Ripple-Carry-Addierer}
–	Volladdierer
–	8-Bit-Addierer
–	Testprotokoll

\subsection{Borrow-Bit-Subtrahierer}
–	Borrow-Subtrahierer
–	8-Bit-Subtrahierer
–	Testprotokoll

\subsection{Ergebnisumschalter}
–	Zwei-zu-eins 1-bit-Umschalter
–	Zwei-zu-eins 8-bit-Umschalter
–	Testprotokoll

\subsection{Binär-zu-BCD-Konverter}
–	Vorbereitungsaufgabe
–	Add3-Block
–	Gesamtschaltung
–	Testprotokoll